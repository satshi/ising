\RequirePackage{plautopatch}
\documentclass[a4paper,12pt,dvipdfmx]{jlreq}
%英語フォント
\usepackage{tgtermes,tgheros,tgcursor}
%日本語を多書体にする
\usepackage{jlreq-deluxe}
%数式用
\usepackage{amsmath}
\usepackage[varg]{newtxmath}
\newcommand{\symbb}{\vvmathbb}
%日本語太字を戻すためのおまじない
\renewcommand{\bfdefault}{bx}
%図のとりこみ
\usepackage{graphicx}

%文献
\usepackage{hep-bibliography}
\bibliography{ref.bib}

\begin{document}

\begin{center}
  \textbf{\sffamily \LARGE 2次元Ising模型のKramers-Wannier双対性}
\end{center}

\begin{flushright}
  \textbf{\sffamily \Large 山口 哲}  
\end{flushright}

\section{導入}
このノートでは2次元Ising模型のKramers-Wannier (KW) 双対性\cite{Kramers:1941kn,Kramers:1941zz}の導出をまとめている。

もちろんKW双対性の導出は少し詳しい目の統計力学の教科書に出ている。それこのノートであらためてまとめておこうと思ったのは2つの理由がある。

1つは熱力学極限だけでなく、精密な双対性について述べたかったからである。もちろんIsing模型の相転移点の決定などに使う場合には熱力学極限だけで十分である。しかし、最近のトポロジカルな議論に使う場合には、離散対称性のゲージ化に関する精密な議論が必要である。ここではそれも含めて導出する。

もう一つは高次元への一般化を紹介したかったからである\cite{Wegner:1971app}。
こちらも最近のトポロジカルな議論での良いモデルになる。

\section{Kramers-Wannier双対性の導出}

\subsection{第1の変形}

\subsection{第2の変形}

\subsection{双対性のまとめ}

\section{双対性からの帰結}

\section{高次元への一般化}

\printbibliography
\end{document}